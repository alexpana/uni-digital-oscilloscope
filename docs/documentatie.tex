\documentclass[10pt,a4paper]{article}
\usepackage[utf8]{inputenc}
\usepackage{amsmath}
\usepackage{amsfonts}
\usepackage{amssymb}
\usepackage[romanian]{babel}
\newcommand{\HRule}{\rule{\linewidth}{0.5mm}}

\begin{document}

\begin{titlepage}
\begin{center}

\textsc{\LARGE Universitatea Tehnica Cluj-Napoca}\\[1.5cm]

\textsc{\Large Facultatea de Automatica si Calculatoare}\\[0.5cm]

% Title
\HRule \\[0.4cm]
{ \huge \bfseries Osciloscop Digital}\\[0.4cm]

\HRule \\[1.5cm]

% Author and supervisor
\begin{minipage}[t]{0.4\textwidth}
\begin{flushleft} \large
\emph{Autori:}\\
% \vspace{3mm}
Alexandru \textsc{Pana}\\
Adrian \textsc{Soucup}\\
gr. 30232
\end{flushleft}
\end{minipage}
\begin{minipage}[t]{0.4\textwidth}
\begin{flushright} \large
\emph{Indrumator:} \\
% \vspace{3mm}
Cristi \textsc{Mocan}
\end{flushright}
\end{minipage}
\vfill

% Bottom of the page
{\large \today}

\end{center}
\end{titlepage}

% TODO
% ---------------------------------------------------------------------------

% Done	Pagina de titlu
% Done	Cuprins (structura documentului anticipata de dumneavoastra)
%  		Introducere (o scurta introducere)
%  		Fundamentare teoretică (pe scurt)
%  		Eventuale Concluzii
% Done 	Bibliografie (numai ce anume credeti dumneavoastra ca va este util)

\tableofcontents

\clearpage

\section{Rezumat}
\clearpage

\section{Introducere}

\subsection{Problema}
\paragraph{}
Osciloscopul este un dispozitiv electronic de testare ce permite 
observarea unor semnale ce variaza constant sub forma unui grafic bidimensional.
Graficul foloseste axa Y pentru a reprezenta valoarea semnalului in functie de timp.

Propunem o implementare in limbaj de descriere hardware a unui osciloscop ce poate
fi implementat pe o placa FPGA ce dispune de un convertor analog digital pentru intrare
si un port VGA pentru iesire.

\section{Fundamentare teoretica}
asdfa sdfa sdf asdf

\section{Proiectare si implementare}

\section{Rezultate experimentale}

\section{Concluzii}

\clearpage

\begin{thebibliography}{99}

\bibitem{lamport94}
  Leslie Lamport,
  \emph{\LaTeX: A Document Preparation System}.
  Addison Wesley, Massachusetts,
  2nd Edition,
  1994.

\end{thebibliography}

\end{document}
