\paragraph{}
Proiectul de fata isi propune implementarea unui osciloscop digital, folosind o placa de dezvoltare Basys2. Implementarea se va face in limbajul VHDL.

\paragraph{}
In acest document exploram notiounea de osciloscop digital, plecand de la definitie, analizand modul de functionare dar si modul de folosire. Vom analiza componentele necesare implementarii unui osciloscop digital: logica de esantionare, o memorie de esantioane si o logica de iesire. Fiecare componenta este explicata din punct de vedere al functionarii, dupa care se prezinta modul de implementare al acesteia specific pentru placuta de dezvoltare Basys2. Pentru logica de esantionare vom explora utilizarea unui convertor analog digital si logica ce controleaza modul de esantionare. Pentru logica de iesire vom explora portul VGA, ce este acesta, cum functioneaza, si cum este implementat pe placuta Basys2.

In descrierea implementarii proiectului avem in considerare tehnicalitatile specifice placutei de dezvoltare cum ar fi constrangerile, limitarile dar si avantajele acesteia. Vom urmari schema osciloscopului si modul de functionare al acestuia.

In sectiunea de rezultate experimentale se analizeaza functionearea proiectului atat in mediu simulat dar si in mediu real. Se pune in functiune placuta Basys2 si se descriu rezultatele obtinute. Scopul aceste sectiuni este de a demonstra corectitudinea implementarii si functionarii proiectului.

Documentul se incheie cu concluziile autorilor, in care acestia prezinta experienta lor pe parcursul dezvoltarii proiectului, notiunile si lectiile invatate, dar si ganduri despre ce ar putea fi imbunatatit sau adaugat.